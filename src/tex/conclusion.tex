%% Макрос для заключения. Совместим со старым стилевиком.
\startconclusionpage

Целью настоящей работы являлось уменьшение временной задержки на передачу данных между процессами в пределах одного физического узла путем разработки и применения методов эффективного межпроцессного взаимодействия. Было выполнено сравнение методов межпроцессного взаимодействия и синхронизации. Исследованы подходы других авторов. Разработаны и реализованы новые методы межпроцессного взаимодействия. Проведено экспериментальное исследование разработанных методов межпроцессного взаимодействия.

Результатом работы стало семейство новых методов межпроцессного взаимодействия. Передача данных в них осуществляется через очередь в разделяемой памяти. Оповещение о появлении данных в очереди осуществляется через разработанный и реализованный мультиплексор оповещений в разделяемой памяти. Он позволяет производить большую часть операций по оповещению процесса-читателя в пользовательской памяти, а ядро использовать только при необходимости процессу-читателю ожидать оповещений в режиме сна и процессу-писателю разбудить его.

Были реализованы различные методы обслуживания соединений по полученным из мультиплексора оповещениям, исследовано влияние активного опроса мультиплексора на временную задержку на передачу данных. Исходя из полученных результатов, разработанные в данной работе методы показали существенно меньшую временную задержку на передачу данных по сравнению с методами, использующими TCP. Метод обслуживания соединений ''Лидер/Последователи`` при использовании с мультиплексором оповещений в разделяемой памяти показывает  меньшую временную задержку на передачу данных, чем метод обслуживания соединений ''Полусинхронный/Полуреактивный``. Применение метода активного опроса мультиплексора позволило существенно уменьшить временную задержку на передачу данных по сравнению с аналогичным методом с пассивным ожиданием оповещений в режиме сна.

Основные результаты работы представлены на IX Конгрессе Молодых Ученых. Результаты работы использованы в программной платформе для высокочастотной алгоритмической торговли Tbricks компании Itiviti.