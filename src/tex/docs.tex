\studygroup{P42111}
\title{Разработка и реализация методов эффективного взаимодействия процессов в распределенных системах}
\author{Губарев Владимир Юрьевич}{Губарев В.Ю.}
\supervisor{Косяков Михаил Сергеевич}{Косяков М.С.}{к.т.н.}{доцент ФПИиКТ}
\publishyear{2020}
%% Дата выдачи задания. Можно не указывать, тогда надо будет заполнить от руки.
%\startdate{01}{сентября}{2017}
%% Срок сдачи студентом работы. Можно не указывать, тогда надо будет заполнить от руки.
%\finishdate{31}{мая}{2019}
%% Дата защиты. Можно не указывать, тогда надо будет заполнить от руки.
%\defencedate{15}{июня}{2019}
%
%\addconsultant{Белашенков Н.Р.}{канд. физ.-мат. наук, без звания}
%\addconsultant{Беззубик В.В.}{без степени, без звания}

\secretary{Болдырева Е.А.}

%% Задание
%%% Техническое задание и исходные данные к работе
\technicalspec{Требуется разработать стилевой файл для системы \LaTeX, позволяющий оформлять бакалаврские работы и магистерские диссертации
на кафедре компьютерных технологий Университета ИТМО. Стилевой файл должен генерировать титульную страницу пояснительной записки,
задание, аннотацию и содержательную часть пояснительной записк. Первые три документа должны максимально близко соответствовать шаблонам документов,
принятым в настоящий момент на кафедре, в то время как содержательная часть должна максимально близко соответствовать ГОСТ~7.0.11-2011
на диссертацию.}

%%% Содержание выпускной квалификационной работы (перечень подлежащих разработке вопросов)
\plannedcontents{Пояснительная записка должна демонстрировать использование наиболее типичных конструкций, возникающих при составлении
пояснительной записки (перечисления, рисунки, таблицы, листинги, псевдокод), при этом должна быть составлена так, что демонстрируется
корректность работы стилевого файла. В частности, записка должна содержать не менее двух приложений (для демонстрации нумерации рисунков и таблиц
по приложениям согласно ГОСТ) и не менее десяти элементов нумерованного перечисления первого уровня вложенности (для демонстрации корректности
используемого при нумерации набора русских букв).}

%%% Исходные материалы и пособия 
\plannedsources{\begin{enumerate}
    \item ГОСТ~7.0.11-2011 <<Диссертация и автореферат диссертации>>;
    \item С.М. Львовский. Набор и верстка в системе \LaTeX;
    \item предыдущий комплект стилевых файлов, использовавшийся на кафедре компьютерных технологий.
\end{enumerate}}

%%% Цель исследования
\researchaim{Разработка удобного стилевого файла \LaTeX
             для бакалавров и магистров кафедры компьютерных технологий.}

%%% Задачи, решаемые в ВКР
\researchtargets{\begin{enumerate}
    \item обеспечение соответствия титульной страницы, задания и аннотации шаблонам, принятым в настоящее время на кафедре;
    \item обеспечение соответствия содержательной части пояснительной записки требованиям ГОСТ~7.0.11-2011 <<Диссертация и автореферат диссертации>>;
    \item обеспечение относительного удобства в использовании~--- указание данных об авторе и научном руководителе один раз и в одном месте, автоматический подсчет числа тех или иных источников.
\end{enumerate}}

%%% Использование современных пакетов компьютерных программ и технологий
\addadvancedsoftware{Пакет \texttt{tabularx} для чуть более продвинутых таблиц}{\ref{sec:tables}, Приложения~\ref{sec:app:1}, \ref{sec:app:2}}
\addadvancedsoftware{Пакет \texttt{biblatex} и программное средство \texttt{biber}}{Список использованных источников}

%%% Краткая характеристика полученных результатов 
\researchsummary{Получился, надо сказать, практически неплохой стилевик. В 2015--2018 годах
его уже использовали некоторые бакалавры и магистры. Надеюсь на продолжение.}

%%% Гранты, полученные при выполнении работы 
\researchfunding{Автор разрабатывал этот стилевик исключительно за свой счет и на
добровольных началах. Однако значительная его часть была бы невозможна, если бы
автор не написал в свое время кандидатскую диссертацию в \LaTeX,
а также не отвечал за формирование кучи научно-технических отчетов по гранту,
известному как <<5-в-100>>, что происходило при государственной финансовой поддержке
ведущих университетов Российской Федерации (субсидия 074-U01).}

%%% Наличие публикаций и выступлений на конференциях по теме выпускной работы
\researchpublications{По теме этой работы я (к счастью!) ничего не публиковал.
\begin{refsection}
Однако покажу, как можно ссылаться на свои публикации из списка литературы:
\nocite{example-english, example-russian}
\printannobibliography
\end{refsection}
}

%% Эта команда генерирует титульный лист и аннотацию.
\maketitle{Магистр}