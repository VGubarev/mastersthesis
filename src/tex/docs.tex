\studygroup{P42111}
\title{Разработка и реализация методов эффективного взаимодействия процессов в распределенных системах}
\author{Губарев Владимир Юрьевич}{Губарев В.Ю.}
\supervisor{Косяков Михаил Сергеевич}{Косяков М.С.}{к.т.н.}{доцент ФПИиКТ}
\publishyear{2020}
%% Дата выдачи задания. Можно не указывать, тогда надо будет заполнить от руки.
\startdate{01}{ноября}{2018}
%% Срок сдачи студентом работы. Можно не указывать, тогда надо будет заполнить от руки.
\finishdate{25}{мая}{2020}
%% Дата защиты. Можно не указывать, тогда надо будет заполнить от руки.
%\defencedate{15}{июня}{2019}
%
%\addconsultant{Белашенков Н.Р.}{канд. физ.-мат. наук, без звания}
%\addconsultant{Беззубик В.В.}{без степени, без звания}

\secretary{Болдырева Е.А.}

\specialty{09.04.04 Программная инженерия}
\specialization{Информационно-вычислительные системы}

\faculty{ПИиКТ}

\university{{\small\bfseries Министерство науки и высшего образования Российской Федерации\par
\MakeUppercase{\scriptsize федеральное государственное автономное образовательное учреждение высшего образования}\par
\begin{singlespace}\MakeUppercase{''Национальный исследовательский университет ИТМО``}\end{singlespace}}}

\programhead{Бессмертный И.А.}{доцент, д.т.н.}

%% Задание
%%% Техническое задание и исходные данные к работе
\technicalspec{Требуется разработать и реализовать эффективные методы межпроцессного взаимодействия в пределах одного физического узла. Межпроцессное взаимодействие как с локальными, так и с удаленными процессами должно осуществляться через единый программный интерфейс. Интерфейс должен автоматически выбирать наиболее эффективный метод межпроцессного взаимодействия и скрывать реализацию от пользователя.}

%%% Содержание выпускной квалификационной работы (перечень подлежащих разработке вопросов)
\plannedcontents{
\begin{enumerate}
\item обзор предметной области и постановка цели работы;
\item разработка и реализация методов эффективного взаимодействия процессов;
\item экспериментальное исследование и обработка результатов.
\end{enumerate}}

%%% Исходные материалы и пособия 
\plannedsources{\begin{enumerate}
\item Косяков М.С. Введение в распределенные вычисления. Учебное пособие / М.С. Косяков. --	СПб: СПбГУ ИТМО, 2014. -- 155 с.
\item Schmidt D.C. et al. Pattern-Oriented Software Architecture, Patterns for Concurrent and Networked Objects. – John Wiley \& Sons, 2013. -- Т. 2.
\end{enumerate}}

%%% Цель исследования
\researchaim{уменьшение временной задержки на передачу данных между процессами в пределах одного физического узла путем разработки и применения методов эффективного межпроцессного взаимодействия.}

%%% Задачи, решаемые в ВКР
\researchtargets{\begin{enumerate}
    \item рассмотреть существующие методы межпроцессного взаимодействия, доступные при взаимодействии процессов, находящихся на одном физическом узле;
    \item произвести анализ и отбор методов межпроцессного взаимодействия для реализации новых методов межпроцессного взаимодействия;
    \item разработать и реализовать эффективные методы межпроцессного взаимодействия;
    \item экспериментально исследовать полученные новые методы межпроцессного взаимодействия.
\end{enumerate}}

%%% Использование современных пакетов компьютерных программ и технологий
\addadvancedsoftware{LaTeX}{Весь текст работы и сопроводительные документы}
\addadvancedsoftware{С++17 (“International Standard ISO/IEC 14882:2017(E) Programming Language C++”)}{Раздел \ref{chapter31}, приложение \ref{sec:app:1}} 
\addadvancedsoftware{LTTng}{Раздел \ref{chapter41}}

%%% Краткая характеристика полученных результатов 
\researchsummary{Разработано семейство новых методов межпроцессного взаимодействия в пределах одного физического узла, показавших существенно меньшую временную задержку на передачу данных, чем методы, использующие TCP.}

\plannedgraphics{
\begin{enumerate}
\item гистограммы временной задержки на передачу данных для разработанных методов межпроцессного взаимодействия;
\item принципиальные схемы разработанных методов межпроцессного взаимодействия.
\end{enumerate}
}

%%% Гранты, полученные при выполнении работы 
\researchfunding{НЕТ}

%%% Наличие публикаций и выступлений на конференциях по теме выпускной работы
\researchpublications{
\begin{enumerate}
\item 
\begin{refsection}
\nocite{GubarevKMU20}
\printannobibliography
\end{refsection}

\item 
\begin{refsection}
\nocite{GubarevFutexConference}
\printannobibliography
\end{refsection}
\end{enumerate}
}

\engabs{
\begin{center}
\small\bfseries Ministry of Science and Higher Education of the Russian Federation\par
\MakeUppercase{\scriptsize FEDERAL STATE AUTONOMOUS EDUCATIONAL INSTITUTION OF HIGHER EDUCATION}\par
\begin{singlespace}\MakeUppercase{''NATIONAL RESEARCH UNIVERSITY ITMO``}\end{singlespace}
\end{center}

\begin{center}\large ABSTRACT\\\small OF A MASTER'S THESIS \end{center}

%Student: Gubarev Vladimir Yurievich

\begin{center}
\large{Development and implementation of efficient inter-process communication methods for distributed systems}
\end{center}

Nowadays distributed systems are widely spreaded. They are usually designed to work in various environments as a set of cooperating processes. At the same time capabilities of modern hardware allow to deploy groups of that processes within a single machine in order to achieve better performance. In this case efficient inter-process communication (IPC) methods become a crucial element of high-performance distributed systems.

The present work is focused on developing efficient IPC methods. Based on the most efficient IPC in Linux, shared memory and futex, it introduces new methods of low-latency IPC. They are transparently provided via a generic interface. The interface automatically and transparently for programmer uses TCP to communicate over network with remote processes and low-latency shared memory-based method for local processes.

Proposed methods show significantly lower latency with local processes than TCP-based without any additional difficulties for programmer.

\vspace{24pt}

\noindent\begin{tabular}{lll}
    Student & Gubarev V.Y. & \signatureplace\\
    Supervisor & Kosyakov M.S. & \signatureplace\\
\end{tabular}

}

%% Эта команда генерирует титульный лист и аннотацию.
\maketitle{Магистр}