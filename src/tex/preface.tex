%% Макрос для введения. Совместим со старым стилевиком.
\startprefacepage

\textbf{Объектом исследования} являются методы межпроцессного взаимодействия.

\textbf{Предметом исследования} является временная задержка на передачу данных между процессами распределенной системы в пределах одного физического узла.

\textbf{Цель работы} -- уменьшение временной задержки на передачу данных между процессами в пределах одного физического узла путем разработки и применения методов эффективного межпроцессного взаимодействия.

В рамках настоящей работы поставлены следующие \textbf{задачи}:

\begin{itemize}
\item рассмотреть существующие методы межпроцессного взаимодействия, доступные при взаимодействии процессов, находящихся на одном физическом узле;
\item произвести анализ и отбор методов межпроцессного взаимодействия для реализации новых методов межпроцессного взаимодействия;
\item разработать и реализовать эффективные методы межпроцессного взаимодействия;
\item эксперементально оценить полученные методы межпроцессного взаимодействия.
\end{itemize}

\textbf{Актуальность исследования}.

Для некоторых систем эффективное межпроцессное взаимодействие является критически важной частью их работы. Требование по минимизации времени обслуживания заявок может напрямую следовать из области применения системы, как в случае с системами для алгоритмической торговли на финансовых рынках. Обслуживание заявок множеством логически связанных процессов может быть существенно ускорено при размещении таких процессов на одном физическом узле. Современные процессоры с количеством ядер более \textbf{40} обеспечивают такую возможность.
Это позволяет использовать более эффективные методы межпроцессного взаимодействия, а именно методы на основе разделяемой памяти \cite{Smith2012DraftH}.
Эффективные методы межпроцессного взаимодействия могут использоваться для связи виртуальных машин или контейнеров в пределах машины-хозяина \cite{IPCInterVirtualMachineShmem, IPCInterVirtualMachineShmemOptimizations}.
Для связи программных модулей, исполняющихся в разных процессах для обеспечения отказоустойчивости за счет изоляции процессов на уровне ОС. 
Для высокопроизводительных вычислений, таких как анализ научных данных или прогнозирование погоды.

При разработке сложной многокомпонентной распределенной системы программисту необходимо сосредоточиться на логике и корректности работы самой системы. В то время как методы межпроцессного взаимодействия должны быть для него прозрачны. Этого можно достичь, используя единый унифицированный интерфейс для межпроцессного взаимодействия. Это упрощает разработку, снимает необходимость сложного управления ресурсами для межпроцессного взаимодействия. А также позволяет автоматически использовать наиболее подходящие методы межпроцессного взаимодействия для данных \textit{пространственных конфигураций} \textbf{(TBD: может, убрать?)} процессов, что может повысить эффективность выполнения некоторых задач этой системой.

Таким образом, разработка и реализация эффективных методов межпроцессного взаимодействия и интерфейса для автоматического доступа к наиболее подходящим из них необходима и обоснована. Посредством этого интерфейса программист прозрачно для себя использует методы межпроцессного взаимодействия на основе разделяемой памяти при взаимодействии с локальными процессами без необходимости перекомпиляции программы. Но поскольку зачастую нельзя разместить всю систему на одном, даже очень производительном, сервере используется TCP при взаимодействии с процессами на других физических узлах.

\textbf{Методы исследования} включают в себя анализ существующих методов межпроцессного взаимодействия, экспериментальную оценку разработанных методов эффективного межпроцессного взаимодействия и методы математической статистики для обработки эксперементальных данных.

\textbf{Средства исследования:}
\begin{itemize}
\item язык программирования C++, компилятор \textit{Clang 6.0.1}, стандартная библиотека C++ \textit{libstdc++};
\item Библиотека Boost.Interprocess \cite{BoostInterprocess} для управления разделяемой памятью;
\item система трассировки событий \cite{LTTngThesis} на основе инструмента LTTng \cite{LTTngSite}.
\end{itemize}

\textbf{Научная новизна} заключается в предложенных новых методах эффективного межпроцессного взаимодействия в пределах одного физического узла, которые не описаны в существующих исследованиях.

\textbf{Положения, выносимые на защиту}

\textbf{TBD.}
\begin{enumerate}
\item shmem + mux + futex + lf;
\item shmem + mux + futex + hsha;
\item shmem + mux + busy wait.
\end{enumerate}


\textbf{Апробация результатов.}

Основные результаты работы были представлены на IX Конгрессе Молодых Ученых.

Предложенные методы межпроцессного взаимодействия на основе разделяемой памяти подготовлены к использованию в платформе для торговли на финансовых рынках Tbricks от компании Itiviti.
