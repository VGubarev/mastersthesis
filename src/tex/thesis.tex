\documentclass[times]{itmo-student-thesis}
% specification,annotation,

%% Опции пакета:
%% - specification - если есть, генерируется задание, иначе не генерируется
%% - annotation - если есть, генерируется аннотация, иначе не генерируется
%% - times - делает все шрифтом Times New Roman, собирается с помощью xelatex
%% - languages={...} - устанавливает перечень используемых языков. По умолчанию это {english,russian}.
%%                     Последний из языков определяет текст основного документа.

%% Делает запятую в формулах более интеллектуальной, например:
%% $1,5x$ будет читаться как полтора икса, а не один запятая пять иксов.
%% Однако если написать $1, 5x$, то все будет как прежде.
\usepackage{icomma}

%% Один из пакетов, позволяющий делать таблицы на всю ширину текста.
\usepackage{tabularx}

\usepackage{courier}

%% Данные пакеты необязательны к использованию в бакалаврских/магистерских
%% Они нужны для иллюстративных целей
%% Начало
\usepackage{tikz}
\usetikzlibrary{arrows}

%% Указываем файл с библиографией.
\addbibresource{master-thesis.bib}

\begin{document}

\studygroup{P42111}
\title{Разработка и реализация методов эффективного взаимодействия процессов в распределенных системах}
\author{Губарев Владимир Юрьевич}{Губарев В.Ю.}
\supervisor{Косяков Михаил Сергеевич}{Косяков М.С.}{к.т.н.}{доцент ФПИиКТ}
\publishyear{2020}
%% Дата выдачи задания. Можно не указывать, тогда надо будет заполнить от руки.
%\startdate{01}{сентября}{2017}
%% Срок сдачи студентом работы. Можно не указывать, тогда надо будет заполнить от руки.
%\finishdate{31}{мая}{2019}
%% Дата защиты. Можно не указывать, тогда надо будет заполнить от руки.
%\defencedate{15}{июня}{2019}
%
%\addconsultant{Белашенков Н.Р.}{канд. физ.-мат. наук, без звания}
%\addconsultant{Беззубик В.В.}{без степени, без звания}

\secretary{Болдырева Е.А.}

\specialty{09.04.04 ''Программная инженерия``}
\specialization{Информационно-вычислительные системы}

\faculty{ПИиКТ}

\university{{\small\bfseries Министерство науки и высшего образования Российской Федерации\par
\MakeUppercase{\scriptsize федеральное государственное автономное образовательное учреждение высшего образования}\par
\begin{singlespace}\MakeUppercase{''Национальный исследовательский университет ИТМО``}\end{singlespace}}}

\programhead{Бессмертный И.А.}{доцент, д.т.н.}

% TBD: не забыть!
\renewcommand{\relatedworksourcecount}{ }

%% Задание
%%% Техническое задание и исходные данные к работе
\technicalspec{Требуется разработать и реализовать эффективные методы межпроцессного взаимодействия в пределах одного физического узла. Межпроцессное взаимодействие как с локальными, так и с удаленными процессами должно осуществляться через единый программный интерфейс. Интерфейс должен автоматически выбирать наиболее эффективный метод межпроцессного взаимодействия и скрывать реализацию от пользователя.}

%%% Содержание выпускной квалификационной работы (перечень подлежащих разработке вопросов)
\plannedcontents{
\begin{enumerate}
\item Обзор предметной области и постановка цели работы.
\item Разработка и реализация методов эффективного взаимодействия процессов.
\item Экспериментальное исследование и обработка результатов.
\end{enumerate}}

%%% Исходные материалы и пособия 
\plannedsources{\begin{enumerate}
\item Косяков М.С. Введение в распределенные вычисления. Учебное пособие / М.С. Косяков. --	СПб: СПбГУ ИТМО, 2014. -- 155 с.
\item Schmidt D.C. et al. Pattern-Oriented Software Architecture, Patterns for Concurrent and Networked Objects. – John Wiley \& Sons, 2013. -- Т. 2.
\end{enumerate}}

%%% Цель исследования
\researchaim{уменьшение временной задержки на передачу данных между процессами в пределах одного физического узла путем разработки и применения методов эффективного межпроцессного взаимодействия.}

%%% Задачи, решаемые в ВКР
\researchtargets{\begin{enumerate}
    \item рассмотреть существующие методы межпроцессного взаимодействия, доступные при взаимодействии процессов, находящихся на одном физическом узле;
    \item произвести анализ и отбор методов межпроцессного взаимодействия для реализации новых методов межпроцессного взаимодействия;
    \item разработать и реализовать эффективные методы межпроцессного взаимодействия;
    \item экспериментально исследовать полученные методы межпроцессного взаимодействия.
\end{enumerate}}

%%% Использование современных пакетов компьютерных программ и технологий
\addadvancedsoftware{LaTeX}{Весь текст диссертации и сопроводительные документы}
\addadvancedsoftware{С++17 (“International Standard ISO/IEC 14882:2014(E) Programming Language C++”)}{Раздел  \ref{chapter31:MuxStructure}, Приложение \ref{sec:app:1}}
\addadvancedsoftware{LTTng}{Глава \ref{chapter41}}

%%% Краткая характеристика полученных результатов 
\researchsummary{Разработано семейство новых методов межпроцессного взаимодействия в пределах одного физического узла, показавших меньшую временную задержку на передачу данных, чем разработанные ранее.}

\plannedgraphics{
\begin{enumerate}
\item Гистограммы временной задержки на передачу данных для разработанных методов межпроцессного взаимодействия.
\item Принципиальные схемы разработанных методов межпроцессного взаимодействия.
\end{enumerate}
}

%%% Гранты, полученные при выполнении работы 
\researchfunding{Отсутствуют.}

%%% Наличие публикаций и выступлений на конференциях по теме выпускной работы
\researchpublications{
\begin{refsection}
\nocite{GubarevFutex}
\printannobibliography
\end{refsection}
}

\engabs{
\begin{center}ABSTRACT

\textbf{Development and implementation of efficient inter-process communication methods for distributed systems}
\end{center}

Nowadays distributed systems are widely spreaded. They are usually designed to work in various environments as a set of cooperating processes. At the same time capabilities of modern hardware allow to deploy groups of that processes within a single machine in order to achieve better performance. In this case efficient inter-process communication (IPC) methods become a crucial element of high-performance distributed systems.

The present work is focused on developing efficient IPC methods. Based on the most efficient IPC in Linux, shared memory and futex, it introduces new methods of low-latency IPC. They are transparently provided via a generic interface. The interface automatically and transparently for programmer uses TCP to communicate over network with remote processes and low-latency shared memory-based method for local processes.

Proposed methods show significantly lower latency with local processes than TCP-based without any additional difficulties for programmer.
}

%% Эта команда генерирует титульный лист и аннотацию.
\maketitle{Магистр}

%% Оглавление
\tableofcontents

%% Макрос для введения. Совместим со старым стилевиком.
\startprefacepage

\textbf{Объектом исследования} являются методы межпроцессного взаимодействия.

\textbf{Предметом исследования} является временная задержка на передачу данных между процессами распределенной системы в пределах одного физического узла.

\textbf{Цель работы} -- уменьшение временной задержки на передачу данных между процессами в пределах одного физического узла путем разработки и применения методов эффективного межпроцессного взаимодействия.

В рамках настоящей работы поставлены следующие \textbf{задачи}:

\begin{itemize}
\item рассмотреть существующие методы межпроцессного взаимодействия, доступные при взаимодействии процессов, находящихся на одном физическом узле;
\item произвести анализ и отбор методов межпроцессного взаимодействия для реализации новых методов межпроцессного взаимодействия;
\item разработать и реализовать эффективные методы межпроцессного взаимодействия;
\item эксперементально оценить полученные методы межпроцессного взаимодействия.
\end{itemize}

\textbf{Актуальность исследования}.

Для некоторых систем эффективное межпроцессное взаимодействие является критически важной частью их работы. Требование по минимизации времени обслуживания заявок может напрямую следовать из области применения системы, как в случае с системами для алгоритмической торговли на финансовых рынках. Обслуживание заявок множеством логически связанных процессов может быть существенно ускорено при размещении таких процессов на одном физическом узле. Современные процессоры с количеством ядер более \textbf{40} обеспечивают такую возможность.
Это позволяет использовать более эффективные методы межпроцессного взаимодействия, а именно методы на основе разделяемой памяти \cite{Smith2012DraftH}.
Эффективные методы межпроцессного взаимодействия могут использоваться для связи виртуальных машин или контейнеров в пределах машины-хозяина \cite{IPCInterVirtualMachineShmem, IPCInterVirtualMachineShmemOptimizations}.
Для связи программных модулей, исполняющихся в разных процессах для обеспечения отказоустойчивости за счет изоляции процессов на уровне ОС. 
Для высокопроизводительных вычислений, таких как анализ научных данных или прогнозирование погоды.

При разработке сложной многокомпонентной распределенной системы программисту необходимо сосредоточиться на логике и корректности работы самой системы. В то время как методы межпроцессного взаимодействия должны быть для него прозрачны. Этого можно достичь, используя единый унифицированный интерфейс для межпроцессного взаимодействия. Это упрощает разработку, снимает необходимость сложного управления ресурсами для межпроцессного взаимодействия. А также позволяет автоматически использовать наиболее подходящие методы межпроцессного взаимодействия для данных \textit{пространственных конфигураций} \textbf{(TBD: может, убрать?)} процессов, что может повысить эффективность выполнения некоторых задач этой системой.

Таким образом, разработка и реализация эффективных методов межпроцессного взаимодействия и интерфейса для автоматического доступа к наиболее подходящим из них необходима и обоснована. Посредством этого интерфейса программист прозрачно для себя использует методы межпроцессного взаимодействия на основе разделяемой памяти при взаимодействии с локальными процессами без необходимости перекомпиляции программы. Но поскольку зачастую нельзя разместить всю систему на одном, даже очень производительном, сервере используется TCP при взаимодействии с процессами на других физических узлах.

\textbf{Методы исследования} включают в себя анализ существующих методов межпроцессного взаимодействия, экспериментальную оценку разработанных методов эффективного межпроцессного взаимодействия и методы математической статистики для обработки эксперементальных данных.

\textbf{Средства исследования:}
\begin{itemize}
\item язык программирования C++, компилятор \textit{Clang 6.0.1}, стандартная библиотека C++ \textit{libstdc++};
\item Библиотека Boost.Interprocess \cite{BoostInterprocess} для управления разделяемой памятью;
\item система трассировки событий \cite{LTTngThesis} на основе инструмента LTTng \cite{LTTngSite}.
\end{itemize}

\textbf{Научная новизна} заключается в предложенных новых методах эффективного межпроцессного взаимодействия в пределах одного физического узла, которые не описаны в существующих исследованиях.

\textbf{Положения, выносимые на защиту}

\textbf{TBD.}
\begin{enumerate}
\item shmem + mux + futex + lf;
\item shmem + mux + futex + hsha;
\item shmem + mux + busy wait.
\end{enumerate}


\textbf{Апробация результатов.}

Основные результаты работы были представлены на IX Конгрессе Молодых Ученых.

Предложенные методы межпроцессного взаимодействия на основе разделяемой памяти подготовлены к использованию в платформе для торговли на финансовых рынках Tbricks от компании Itiviti.


\chapter{Обзор предметной области и постановка цели работы}

\startrelatedwork

\section{Методы межпроцессного взаимодействия}

Метод межпроцессного взаимодействия -- это способ осуществления взаимодействия процессов, находящихся на одном или разных физических узлах.

В Linux поддержаны следующие методы межпроцессного взаимодействия:
\begin{enumerate}
\item Интернет-сокеты и сокеты домена Unix.
\item именованные и неименованные каналы.
\item Очередь сообщений.
\item Разделяемая память: SystemV и POSIX.
\end{enumerate}

Рассмотрим приведенные выше методы межпроцессного взаимодействия.

\subsection{Сокеты}

Сокет -- это программный интерфейс для обеспечения обмена данными между процессами. В зависимости от реализации интерфейса, позволяет взаимодействовать процессам как на разных физических узлах в составе сети, так и в пределах одного узла. Широко распространены два вида, сокеты TCP и сокеты домена Unix. Первый работает через протокол TCP, второй -- использует некоторую внутреннюю реализацию канала связи, как и в TCP.

\subsubsection{TCP-сокеты}
TCP-сокет -- интерфейс межпроцессного взаимодействия, использующий протокол TCP. Среди всех рассматриваемых методов только он позволяет взаимодействовать процессам на разных физических узлах в составе сети. В то же время, возможно межпроцессное взаимодействие и в пределах одного узла через механизм обратной петли, когда TCP-сообщение передается без каких-либо излишних операций в процесс-получатель, не покидая физического узла. В качестве точки соединения используется пара IP-адрес и порт.

TCP широко распространен. Сторонние системы чаще всего предоставляют именно TCP-интерфейс доступа к своим службам.

\subsubsection{Сокеты домена Unix}

Unix-сокет или IPC-сокет -- интерфейс межпроцессного взаимодействия в пределах одного узла, не использующий сетевого протокола. Может работать в разных режимах: передачи потока байт, датаграмм или последовательных пакетов. Первые два соответствуют протоколам TCP и UDP. Третий -- последовательный надежный канал для передачи датаграмм.

\subsection{Каналы}

\subsubsection{Неименованные каналы}

Неименованный канал -- однонаправленный метод межпроцессного взаимодействия для родственных процессов. Данные, записанные в неименованный канал, остаются там до момента считывания, либо до момента завершения ссылающихся на него процессов. Размер буфера ограничен.

\subsubsection{Именованные каналы}

Именованный канал отличается от неименованного тем, что представлен в виде файла, следовательно, доступен всем процессам ОС. Кроме того, время его жизни не ограничено временем жизни использующих его процессов.

\subsection{Очередь сообщений}

Очередь сообщений -- список сообщений, который хранится в пространстве ядра. Каждая очередь имеет свой уникальный идентификатор. Взаимодействие происходит путем вызовов записи и чтения сообщений.

\subsection{Разделяемая память}

Обычно в целях безопасности адресное пространство процесса изолировано от других. В некоторых случаях, однако, может быть необходимо использовать совместно один и тот же сегмент памяти. Такая память называется разделяемой. На рисунке \ref{chapter11:ShMem} приведен пример использования разделяемой памяти двумя процессами.

\begin{figure}[!h]
\caption{Пример использования разделяемой памяти}
\label{chapter11:ShMem}
%\includegraphics{../../graphics/schemes/ShMemBasic}
\tikzfig{../../graphics/schemes/ShMemBasic}
\end{figure}

Существуют два интерфейса для доступа к разделяемой памяти. Более старый System V и более новый POSIX.

\subsubsection{System V}

Разделяемая память здесь является системным ресурсом, она представлена уникальным ключом в пределах ОС.

Интерфейс реализуется через набор системных вызовов и структур для работы с ним:
\begin{itemize}
\item shmget(key\_t key, size\_t size, int oflag) -- вызов для создания нового сегмента разделяемой памяти или использования существующего с ключом \textit{key} . Размер сегмента \textit{size}, флагb доступа и создания \textit{oflag};
\item shmat(int shmid, const void *shmaddr, int flag) -- подключение сегмента в адресное пространство процесса;
\item shmdt(const void *shmaddr) -- отключение сегмента от адресного пространства;
\item shmctl(int shmid, int cmd, struct shmid\_ds *buf) -- управление разделяемой памятью: изменение прав доступа, удаление, запрос статистики;
\end{itemize}

\subsubsection{POSIX}

Разделяемая память здесь является пользовательским ресурсом, она представлена файлом.
Интерфейс реализуется через набор системных вызовов:
\begin{itemize}
\item shm\_open(const char *name, int oflag, mode\_t mode) -- открывает файл для разделяемой памяти (аналог shmget). В отличие от обычного вызова open, открытый таким образом файл не синхронизируется с диском.
\item shm\_unlink(const char *name) -- удаляет файл (аналог shmctl);
\item ftruncate(int fd, off\_t length) -- задает размер файла;
\item fstat(int fd, struct stat *statbuf) -- статистика о файле (аналог shmctl);
\item mmap(void *addr, size\_t length, int prot, int flags, int fd, off\_t offset) -- отображение файла в памяти (аналог shmat);
\item munmap(void *addr, size\_t length) -- отключает отображенный сегмент от адресного пространства процесса (аналог shmdt).
\end{itemize}

\subsubsection{Разница между System V и POSIX}
Оба интерфейса предлагают аналогичные возможности. Разница состоит в предоставлении ресурса в системе, имени файла для POSIX или целом числе в System V. Также, в POSIX интерфейсе сегмент разделяемой памяти будет уничтожен, когда будет завершен последний процесс, отображающий его в память, а файл полностью удален. Это свойство полезно на случай непредвиденного завершения процессов, например, в результате программного дефекта.

\subsection{Сравнение методов межпроцессного взаимодействия}

Приведенные выше методы можно разделить на группы по трем критериям:
\begin{itemize} 
\item способности к взаимодействию с удаленными процессами, то есть с процессами на других физических узлах;
\item использованию ядра ОС для осуществления межпроцессного взаимодействия.
\end{itemize}

Среди всех представленных методов только TCP-сокеты позволяют взаимодействовать как процессам на разных физических узлах, так и на одном узле, используя один и тот же интерфейс. Поэтому разумно использовать именно этот метод в качестве базового метода межпроцессного взаимодействия.

Все приведенные методы, за исключением разделяемой памяти, при межпроцессном взаимодействии используют ядро операционной системы. Использование ядра существенно увеличивает временную задержку на передачу данных. Как минимум, из-за двойного копирования данных, перехода между пользовательским режимом и режимом ядра (см. рисунок \ref{chapter11:KernelUsage}).
Разделяемая память же позволяет осуществлять взаимодействие напрямую через пользовательское пространство.

\begin{figure}[!h]
\caption{Общая схема межпроцессного взаимодействия при использовании ядра ОС}
\label{chapter11:KernelUsage}
\tikzfig{../../graphics/schemes/KernelUsage}
\end{figure}

Однако, сама по себе разделяемая не предоставляет возможностей для синхронизации процессов, не имеет механизма оповещения об изменении состояния разделяемой памяти. Для этого необходимо рассмотреть методы синхронизации процессов.

\section{Методы синхронизации процессов}

Методы синхронизации процессов нужны, чтобы корректно передавать данные между процессами через разделяемую память, например, не допускать состояния гонки. Можно выделить два класса методов: примитивы синхронизации и атомарные операции. Примитивы синхронизации зачастую используют в своих алгоритмах атомарные операции.

\subsection{Примитивы синхронизации}

В Linux на уровне ядра ОС поддержаны следующие примитивы межпроцессной синхронизации:
\begin{enumerate}
\item System V семафоры;
\item futex -- \textbf{f}ast \textbf{u}serspace mu\textbf{tex}.
\end{enumerate}

В свою очередь, futex служит основой для более продвинутых примитивов синхронизации:
\begin{enumerate}
\item POSIX-семафора;
\item mutex -- взаимное исключение;
\item rw-mutex -- взаимное исключение для одного писателя и множества читателей.
\item spinlock -- взаимное исключение методом холостого ожидания (busy wait).
\item Условная переменная -- оповещение о наступлении события и ожидание события.
\end{enumerate}

Рассмотрим их подробнее.

\subsubsection{futex}

Futex \cite{FutexOrigins} -- это низкоуровневый механизм блокировок в пользовательском пространстве. В основном, операции с ним производится с помощью атомарных инструкций в пользовательском пространстве. Ядро применяется для ожидания и диспетчеризации использующих futex процессов.

В пользовательской памяти futex -- это 4-байтное число, выравненное также на 4 байта. Он может располагаться в собственном адресном пространстве процесса и использоваться для синхронизации потоков процесса, а может и в разделяемой памяти и отвечать за синхронизацию разных процессов в пределах одного физического узла. Алгоритм работы с ним устанавливается пользователем. В нужных сценариях алгоритма используется ядро ОС для разрешения состязательности, задач ожидания и диспетчеризации.

\subsubsection{Семафоры}

Семафор -- это примитив синхронизации на основе целочисленного счетчика. Поддерживает две операции: увеличение значения счетчика и уменьшение. Уменьшение до нуля приводит к добавлению процесса в очередь ожидания семафора. Увеличение от нуля -- к пробуждению одного из потоков из очереди ожидания.

В Linux представлены два интерфейса семафоров: System V и POSIX.

\paragraph{POSIX}

Реализованы через futex. Представляют из себя целое знаковое число. Процедуры \textit{sem\_post()} и \textit{sem\_wait()} используются для увеличения и уменьшения значения семафора, соответственно. Первая также пробуждает процессы в очереди ожидания семафора, если необходимо, вторая -- добавляет процесс в очередь ожидания, если после уменьшения значения семафора стало равно нулю или было равно  нулю.
POSIX-семафоры представлены в двух видах: именованные и неименованные. Первые идентифицируются файлом, вторые располагаются в адресном пространстве процесса или разделяемой памяти.

\paragraph{System V}

Реализованы в ядре ОС. Идентифицируются целочисленным ключом в пределах одной ОС. Имеют возможность контролировать доступ к операциям над семафором.

\subsubsection{Взаимные исключения}

Взаимное исключение -- примитив синхронизации для обеспечения эксклюзивного доступа к данным, защищенным этим примитивом. Существует несколько вариантов: обычный -- mutex, рекурсивный -- recursive mutex, с одновременным доступом нескольких читателей -- read/write-mutex. По определению, mutex может освободить только тот поток, который его занял.

В Linux реализованы через futex. Предоставляются посредством библиотеки \textit{pthread}.

\paragraph{Mutex}
Классический механизм взаимного исключения. Представлен платформо-зависимым типом данных \textit{pthread\_mutex\_t} Используется через семейство функций библиотеки \textit{pthread} -- \textit{pthread\_mutex\_*}.

\paragraph{RW-Mutex}
Версия взаимного исключения, допускающая совместное нахождение в критической секции нескольких читателей, но требующая эксклюзивного доступа для писателя. Представлена платформо-зависимым типом данных \textit{pthread\_rwlock\_t} Используется через семейство функций библиотеки \textit{pthread} -- \textit{pthread\_rwlock\_*}.

\paragraph{Spinlock}
Реализация механизма взаимного исключение с использованием механизма холостого ожидания. Взятие блокировки состоит в изменение значения переменной на условные "свободно" и "занято", а ожидание -- в непрерывном циклическом ожидании наступления состояния "свободно".

\subsubsection{Условная переменная}

Условная переменная -- примитив синхронизации. Обеспечивает блокирование одного или нескольких потоков до наступления события о выполнения предиката. Представлена платформо-зависимым типом данных \textit{pthread\_cond\_t}. Используется через семейство функций библиотеки \textit{pthread} -- \textit{pthread\_cond\_*}. Условная переменная используется вместе с mutex, который обеспечивает синхронизированный доступ к параметрам предиката и другим данным.

В Linux примитив также реализован через futex.

\subsection{Сравнение методов межпроцессной синхронизации}

\textbf{TBD: прямо сравнение? Выбрать нормальное название раздела}

Наиболее универсальный из рассмотренных классических примитивов синхронизации -- это семафор. Используя его можно реализовать как взаимное исключение, так и условную переменную. Он подходит для решения задачи читателей и писателя (или производителя и потребителей), как и комбинация условной переменной с mutex.

Как можно заметить, почти все рассмотренные примитивы в Linux реализованы через futex. Он может быть полезен при разработке своих собственных специализированных примитивов синхронизации \cite{FutexesAreTricky}.

\subsection{Атомарные операции}

Атомарная операция -- это операция, которая выполняется исключительно целиком, либо не выполняется вовсе. В настоящей работе рассматриваются именно простейшие операции. Они используются через стандартную библиотеку языка C++ как набор классов \textit{std::atomic}.

Виды атомарных операций:
\begin{enumerate}
\item load/store -- атомарные чтение или запись;
\item swap -- атомарно устанавливает значение переменной и возвращает старое значение;
\item compare-and-swap -- атомарное изменение значения переменной, если ее текущее состояние совпадает с ожидаемым;
\item fetch-and-(add/dec/or/and) -- атомарная арифметическая или логическая операция над переменной с возвращением предыдущего значения.
\end{enumerate}

Атомарные операции используются при разработке методов синхронизации и неблокирующих алгоритмов, структур данных с конкурентным доступом. Даже в простейшей реализации взаимного исключения, в spinlock, исполнение в критической секции требует двух атомарных операций на взятие и освобождение. Если внутри критической секции необходимо, например, выставить бит в 4-байтной переменной, то можно сделать это, используя атомарную операцию fetch-and-or. 

Вместе с futex они являются наиболее низкоуровневыми блоками для построения методов межпроцессной синхронизации.

\section{Предыдущая работа}

В ранних работах автора была разработана и реализована прикладная программная платформа для эффективного взаимодействия процессов \cite{GubarevKMU18} по модели ''точка-точка``. Разработан и реализован метод эффективного межпроцессного взаимодействия посредством мультиплексора в разделяемой памяти \cite{GubarevKMU20}. Этот метод предоставляется через реализованную ранее прикладную программную платформу, то есть не требует со стороны программиста никаких изменений в программном коде для использования разработанного метода.
\textbf{TBD: может, убрать все цитирования ВКР? А еще мультиплексор -- это не предыдущая, а очень даже свежая работа.}

\section{Обзор литературы}

\subsection{Научные работы}
\textbf{TBD: подумать над названием раздела}
Во множестве работ предлагаются методы эффективного межпроцессного взаимодействия, системы передачи сообщений между процессами в различных применениях. В подавляющем большинстве работ в качестве среды для передачи данных используется разделяемая память.

Авторы в работе \cite{7790679} предлагают гетерогенную систему для передачи сообщений по модели ''издатель/подписчики``. Посредством очереди разделяемой памяти данные передаются между процессами, написанными на разных языка: Python и C++. Синхронизация и ожидание данных осуществляется через семафор.

В работе \cite{ShMemTransportThesis} представлена система для передачи сообщений, использующая разделяемую память для взаимодействия процессов в пределах одного узла и TCP -- с удаленными процессами. Поддержаны взаимодействия как по модели ''точка-точка``, так и ``издатель/подписчики''. Система предоставляет единый интерфейс межпроцессного взаимодействия, то есть использование того или иного метода не требует изменения исходного кода программистом.

В параллельных вычислениях широко используется MPI -- Message Passing Interface. Это программный интерфейс для взаимодействия процессов посредством передачи сообщений. В ряде работ предлагаются различные методы межпроцессного взаимодействия на основе разделяемой памяти с целью увеличения производительности параллельных систем в различных реализациях MPI, например, MPICH2 \cite{10.1007/11846802_19, 8665755} или MVAPICH2 \cite{4100356, 4663761}.

В работе \cite{HPCInterCoreLinux} предлагается метод эффективного межпроцессного взаимодействия с использованием разделяемой памяти для встраиваемых многоядерных систем, где необходимы одновременно производительные и не ресурсоемкие методы.

Существуют работы об использовании разделяемой памяти для взаимодействия процессов в разных виртуальных машинах в пределах машины-хозяина \cite{IPCInterVirtualMachineShmem, IPCInterVirtualMachineShmemOptimizations, 10.1145/2847562, 10.1007/978-3-642-00955-6_7}. В том числе работа \cite{234795}, в которой разделяемая память используется как для взаимодействия процессов на одном физическом узле, так и на разных в пределах кластера через удаленный прямой доступ в память (RDMA), что требует специального оборудования. В работе \cite{7416013} авторы используют разделяемую память для повышения производительности взаимодействия виртуальных машин на одном физическом узле через сокеты (TCP и UDP).

Эффективное межпроцессное взаимодействие также важно в роботизированных системах. Так, в работе \cite{5649358} предлагается система передачи сообщений по модели ''издатель/подписчики`` посредством UDP-мультивещания с целью добиться минимальной временной задержки на передачу данных. Метод межпроцессного взаимодействия через разделяемую память предлагается в работе \cite{Wang2019TZCEI}, где авторы предлагают систему ''издатель/подписчик`` с передачей управляющего блока данных через сокеты, а сами данные через разделяемую память без преобразования сериализации передаваемого объекта.

В различных работах исследуется производительность механизмов операционных систем для межпроцессного взаимодействия. В работе \cite{MuxComparison} сравниваются системные мультиплексоры событий \textit{select, epoll и poll}. В работе \cite{AnalysisOfEventProcDesignPatterns} проведен анализ влияния специфики работы этих механизмов на производительность методов обслуживания соединений ''Лидер/Последователи`` и ``Полусинхронный/Полуасинхронный``. А в работе \cite{6298185} были предложены марковские модели этих методов обслуживания.
\textbf{TBD: нужны ли мне модели тут?}
%В работах \cite{Smith2012DraftH, venkataraman2015evaluation} произведен анализ и сравнение различных методов межпроцессного взаимодействия в Linux: каналов, различных реализаций на основе разделяемой памяти, сокетов домена UNIX и TCP.
В работах \cite{Smith2012DraftH, venkataraman2015evaluation} в Linux показано превосходство методов межпроцессного взаимодействия на основе разделяемой памяти над каналами, TCP и сокетами домена Unix.
В работе \cite{8526899} исследуются механизмы пробуждения потоков для обслуживания асинхронных событий и предлагается метод существенного уменьшения временной задержки на пробуждение потока для обслуживания асинхронного события, используя привилегированные возможности аппаратного обеспечения.

\subsection{Коммерческие решения для передачи сообщений}

Наиболее известные высокопроизводительные решения для передачи сообщений между процессами -- это системы от компаний TIBCO \cite{TIBCO}, Informatica \cite{Informatica} и Solace \cite{Solace}. Они представляют собой целостные решения организации обмена сообщениями между процессами распределенной системы. А для передачи данных между процессами в пределах одного физического узла используют методы на основе разделяемой памяти.
\textbf{TBD: а надо ли оно мне?}

\section{Значимость методов на основе разделяемой памяти в межпроцессном взаимодействия}

Исходя из выше перечисленного, методы на основе разделяемой памяти имеют большое значение в современных системах обмена сообщениями. Они позволяют передавать сообщения между процессами с минимальной временной задержкой.

\section{Важность единого интерфейса межпроцессного взаимодействия}

Крайне важно соблюдать единый и неизменный интерфейс межпроцессного взаимодействия. Внедрение новых методов межпроцессного взаимодействия не должно приводить к изменениям в пользовательском исходном коде. Поэтому методы, разрабатываемые в настоящей работе, должны соблюдать ранее представленный интерфейс межпроцессного взаимодействия \cite{GubarevKMU18}.

\finishrelatedwork

\section{Критерий эффективности и постановка цели}

В настоящей работе критерием эффективности принята временная задержка на передачу данных. Чем она меньше, тем более эффективным считается метод межпроцессного в взаимодействия.
Цель работы -- уменьшение временной задержки на передачу данных между процессами в пределах одного физического узла по сравнению с TCP путем разработки и применения более эффективных методов межпроцессного взаимодействия.

Чтобы уменьшить временную задержку на передачу данных, необходимо разработать и реализовать новые, более эффективные методы межпроцессного взаимодействия, чем представленные автором \textbf{TBD: автором?} ранее. Они должны быть построены на основе очереди в разделяемой памяти и использовать более эффективные методы оповещения о появлении данных в этой разделяемой памяти, чем TCP.

Из проведенного сравнения методов межпроцессного взаимодействия и синхронизации и обзора литературы следует, что сама по себе разделяемая память является очень производительным методом межпроцессного взаимодействия. Однако, она не предоставляет механизмов синхронизации и отслеживания событий в ней. Поэтому необходимо также разработать и реализовать эффективный метод оповещения о появлении данных в разделяемой памяти.

Необходимо экспериментально исследовать разработанные методы межпроцессного взаимодействия и сравнить их с предложенным ранее методом с использованием разделяемой памяти и TCP для оповещения о наличии данных в очереди.

Новые методы должны следовать разработанному ранее интерфейсу межпроцессного взаимодействия и не требовать изменений в пользовательском коде.

\chapter{Анализ методов межпроцессного взаимодействия}

\chapter{Анализ методов межпроцессного взаимодействия}

\chapterconclusion

В конце каждой главы желательно делать выводы. Вывод по данной главе~--- нумерация работает корректно, ура!

\chapter{Разработка и реализация методов эффективного взаимодействия процессов}
%
%\section{Интерфейс доступа к разрабатываемым методам межпроцессного взаимодействия}
%
%\textbf{TBD: можно взять из бакалаврской ВКР или оставить на нее ссылку.}
%
%Существенная часть межпроцессного взаимодействия - это то, как программист его использует и как сделать это удобным процессом. Интерфейс сокетов в ОС Linux хорошо для этого, так как использует механизм файловых дескрипторов
%\textbf{TBD: определение или ссылку}
%, а значит:
%\begin{itemize}
%\item имеет стандартные примитивы чтения и записи: системные вызовы \textit{read/readv} и \textit{write/writev};
%\item позволяет мультиплексировать множество соединений для одновременного отслеживания посредством системных мультиплексоров: \textit{select/poll/epoll}.
%\end{itemize}
%
%Удобный и унифицированный интерфейс может объединять под собой совершенно разные методы межпроцессного взаимодействия прозрачным для программиста образом. Программист зачастую не заинтересован в тонкостях передачи данных, ему необходимо реализовывать и совершенствовать пользовательскую логику приложения, однако, поскольку процессы распределенной системы работают вместе над одними задачами, то ему важна и эффективность межпроцессного взаимодействия.
%
%Необходима возможность:
%\begin{itemize}
%\item предоставлять другим процессам возможность для подключения и подключаться к другим процессам;
%\item принимать и передавать данные.
%\end{itemize}
%
%Для первого пункта хорошо подходит семейство шаблонов сетевого программирования Acceptor и Connector \cite{schmidt1996acceptor}, которые при необходимости создают экземпляры пользовательских обработчиков соединений, реализующих интерфейс, представленный на листинге \ref{chapter31:ServiceHandlerInterface}.
%
%\begin{lstlisting}[float=!h,caption={Интерфейс пользовательского обработчика соединений на C++},label={chapter31:ServiceHandlerInterface}]
%class ISession
%{
%public:
%	virtual int handle_message(const Message & message) = 0;
%	virtual int send(const Message & message) = 0;
%	virtual int handle_close() = 0;
%};
%\end{lstlisting}
%
%%Поскольку существует необходимость полностью оградить программиста от тонкостей работы межпроцессного взаимодействия, чтобы упростить работу пользовательского кода и иметь возможность выбирать наиболее подходящий для данной ситуации метод межпроцессного взаимодействия. выбран событийный асинхронный метод доставки сообщений. Для этого в интерфейсе пользовательского обработчика соединения предусмотрен метод \textit{handle\_message(const Message \&)}. Нижележащая реализация метода межпроцессного взаимодействия вызывает данный метод для соответствующего соединения по факту получения ей очередного сообщения. Для отправки сообщений программисту предлагается использовать примитив \textit{send(const Message \&)}.
%
%\textbf{TBD: кривовато звучит}
%Данный интерфейс соответствует асинхронному событийному шаблону проектирования приложений. В этом шаблоне обработчик соединения является пассивной сущностью, в которую события доставляются посредством вызова метода \textit{handle\_message(const Message \&)}. Во-первых, использование этого шаблона позволяет разделить пользовательскую логику и нижележащие методы межпроцессного взаимодействия, работающие по произвольным алгоритмам. Следовательно, это упрощает пользовательскую логику. Во-вторых, позволяет обрабатывать множество соединений числом потоков, не равным количеству этих соединений, так как обработчики соединения нуждаются в потоке для своего выполнения только когда необходимо обработать событие. Для отправки сообщений программисту предлагается использовать примитив \textit{send(const Message \&)}.
%
%Данный интерфейс позволяет скрыть от программиста тонкости работы межпроцессного взаимодействия. Например, отслеживание готовности множества TCP-соединений к передаче или приему данных, состояние соединений. Или то, что взаимодействие происходит через разделяемую память.
%
%\section{Базовый метод межпроцессного взаимодействия на основе TCP}\label{chapter31:PureTCP}
%
%TCP -- это наиболее широко-используемый и универсальный метод межпроцессного взаимодействия. Протокол гарантирует, что сообщения будут получены в том же количестве и порядке, в котором они были отправлены. Посредством интерфейса сокетов он позволяет взаимодействовать как с локальными, так и с удаленными процессами распределенной системы. Более того, сторонние системы тоже зачастую предоставляют возможность именно подключения по TCP.
%
%Кроме распространенности и удобства для программиста, предоставляют очень важную возможность отслеживания времени жизни соединения. Это важно, так как случае краха процесса операционная система в ходе освобождения системных ресурсов также закроет все TCP-соединения процесса и другие стороны межпроцессного взаимодействия смогут корректно обработать закрытие соединений.
%
%\subsection{Мультиплексирование соединений}
%
%Межпроцессное взаимодействие посредством TCP позволяет использовать системные мультиплексоры оповещений \textit{select/poll/epoll} для отслеживания оповещений в множестве TCP-соединений: наличия данных, готовности канала к передаче, закрытия соединения. В системах с большим количеством неактивных соединений лучше всего себя показывает системный мультиплексор \textit{epoll}, при наличии небольшого количества активных соединений большой разницы между ними не наблюдается \cite{MuxComparison}.
%
%Классическим подходом при разработке сетевых приложений является применение шаблона Реактор \cite{schmidt1995reactor, 10.1145/1808954.1808964} для диспетчеризации множества соединений. В настоящей работе используется реализаций шаблона Реактор из библиотеке ACE \cite{ACE}, работающая с системным мультиплексором оповещений \textit{select}.
%
%Реактор -- это объект, в котором регистрируются обработчики событий и которые Реактор вызывает посредством вызовов их интерфейсных методов. Сокращенный интерфейс обработчика событий из библиотеки ACE приведен в листинге \ref{chapter31:LowLevelServiceHandlerInterface}.
%\begin{lstlisting}[float=!h,caption={Интерфейс низкоуровневого обработчика соединений на C++},label={chapter31:LowLevelServiceHandlerInterface}]
%class ACE_Event_Handler
%{
%public:
%	virtual int handle_input(ACE_HANDLE fd) = 0;
%	virtual int handle_output(ACE_HANDLE fd) = 0;
%	virtual int handle_close(ACE_Handle fd, ACE_Reactor_Mask close_mask) = 0;
%};
%\end{lstlisting}
%
%\textbf{TBD}
%
%\subsection{Обслуживание активных соединений}
%
%\textbf{TBD}
%
%\subsection{Динамическое конфигурирование соединений}
%\textbf{TBD: надо написать про стек модулей}
%
%
%\section{Применение разделяемой памяти для передачи данных}
%
%\textbf{TBD: описать SPSC очередь на буфере постоянного размера}
%
%Однако, недостаточно разместить очередь в разделяемой памяти и отправлять в нее сообщения. Кроме контроля времени жизни соединения, существенный вопрос в данном методе: кто, как и когда будет принимать сообщения. На одном физическом узле может находится множество взаимодействующих процессов с большим количеством соединений между ними. Соответственно, для каждого соединения появляется своя точка синхронизации, в которой процесс-читатель может узнать, что очередь не пуста.
%
%Необходимо разработать метод оповещения процесса-читателя о появлении данных в разделяемой памяти, которые ему необходимо принять и обработать.

\section{Методы оповещения о появлении данных в разделяемой памяти}

\subsection{Наивные алгоритмы в разделяемой памяти}

Процесс-читатель знает о расположении всех очередей в разделяемой памяти, в которые отправляют сообщения все процессы-писатели, с которыми он взаимодействует. Непосредственно само состояние очередей может быть использовано для оповещения процесса-читателя о наличии данных в этих очередях.

\paragraph{Алгоритм №1}

При небольшом количестве соединений (например, $0.25 * N$, где $N$ - количество ядер в процессоре) возможно использование выделенных потоков в процессе-читателе для активного опроса состояния очереди и обслуживания соответствующего соединения.  \textbf{TBD: иллюстрацию?}

\paragraph{Алгоритм №2}

При большем количестве соединений возможно использовать группу выделенных потоков, активно опрашивающих которое количество очередей в разделяемой памяти. Например, 1 поток, активно опрашивающий до 10 соединений.
\textbf{TBD: иллюстрацию?}

\subsubsection{Применимость, достоинства и недостатки}\label{chapter31:NaivePolling}

Данные алгоритмы активного опроса очередей в разделяемой памяти для обслуживания соединений вполне могут быть использованы в реальных системах. Их можно применить в системах: с большими вычислительными ресурсами, небольшим количеством процессов и очень активных соединений.

В противном случае, активно работающие потоки будут выполнять много бесполезных операций по опросу пустых очередей неактивных соединений. Кроме того, постоянно работающий поток, опрашивающий очереди в разделяемой памяти, может быть вытеснен с процессора планировщиком операционной системы из-за израсходования отведенного ему кванта процессорного времени, что ухудшит качество обслуживания заявок.

Приведенные методы в настоящей работе не рассматриваются, поскольку количество соединений между процессами на одном физическом узле и самих процессов может быть большим, а опрос очередей на наличие в них данных сопровождается взятием взаимной блокировки, что приводит к неэффективному использованию аппаратных ресурсов.

\subsection{TCP}\label{chapter31:SignalTCP}

Как было сказано выше, TCP используется как базовый метод межпроцессного взаимодействия. Он может быть использован и как метод оповещения о появлении данных в очереди в разделяемой памяти.

\subsubsection{Алгоритм взаимодействия при использовании TCP для оповещения о появлении данных}

\textbf{TBD: нужна иллюстрация взаимодействия и стек модулей}
% Чтобы процесс-читатель о наличии данных в очереди в разделяемой памяти, необходимо:

\begin{enumerate}
\item процессу-читатель ожидает новых данных по всем своим TCP соединениям в состоянии сна в системном мультиплексоре оповещений;
\item для передачи данных таким методом процессу-писателю необходимо записать в очередь нужное сообщение и передать на нижележащий модуль сообщение минимального размера в 1 байт с заранее установленным значением (например, ''0``);
\item ядро операционной системы пробуждает процесс-читатель;
\item реактор процесса-читателя демультиплексирует активное TCP соединение, считывает 1 байт полезных данных и отправляет его на следующий слой обработки межпроцессных взаимодействий через ранее описанный стек модулей;
\item модуль, отвечающий за взаимодействие по разделяемой памяти, проверяет, что полученный 1 байт имеет заранее оговоренное значение (''0`` в примере выше) и это служит для него сигналом к проверке состояния очереди в разделяемой памяти для соединения, с которого этот сигнальный байт был получен;
\item процесс-читатель считывает сообщение из очереди и выполняет его обработку.
\end{enumerate}

Таким образом происходит передача данных между процессами в разделяемой памяти с оповещением о появлении данных в разделяемой памяти по TCP.

\subsubsection{Работа с очередью в разделяемой памяти при использовании TCP для оповещения о появлении данных}\label{chapter31:SharedMemoryOptimization}

\textbf{TBD: module stack, handshake}
\textbf{TBD: нужна иллюстрация взаимодействия или какой- нибудь псевдокод}

\subsubsection{Достоинства и недостатки}

Предложенный метод обладает следующими \textbf{достоинствами}:
\begin{itemize}
\item позволяет эффективно поддерживать множество соединений с использованием системных мультиплексоров оповещений (\textit{select/poll/epoll});
\item позволяет процессу-читателю блокировать свое выполнение до появления оповещения;
\item благодаря использованию эвристики из раздела \ref{chapter31:SharedMemoryOptimization} временная задержка на передачу данных может быть снижена, если к концу обработки очередного сообщения в очередь уже будет записано новое сообщение;
\item метод межпроцессного взаимодействия по TCP не требует доработки и может быть использован как есть.
\end{itemize}

\textbf{Недостаток} у такого подхода только один: временная задержка на отправку и получение оповещения. Основные отличия от метода, использующего только TCP, в том, как передается само сообщения. Но кроме использования среды для передачи самих данных выполняется рад системных вызовов для оповещения процесса-читателя, каждый из которых может вносить существенную временную задержку:
\begin{itemize}
\item \textit{write} -- для записи 1 байта в TCP-сокет процессом-писателем;
\item \textit{select/poll/epoll} -- для демультиплексирования нужного оповещения среди множества источников процессом-читателем.
\item \textit{read} -- для чтения 1 байта из TCP-сокета процессом-читателем.
\end{itemize}

В случае, когда процесс-читатель находится в состоянии сна на системном мультиплексоре оповещений, процесс-писатель в ходе системного вызова \textit{write} должен также изменить состояние процесса-читателя на ''Готов к выполнению`` или ''Выполняется``. После этого в течение некоторого промежутка времени процесс-читатель будет готовиться к выполнению. Все это может влиять на временную задержку на передачу данных.

Следовательно, необходимо разработать новый метод мультиплексирования соединений, использующих разделяемую память, имеющий меньшие накладные расходы на использование и обладающий, как минимум, теми же достоинствами, что и описанный в данном разделе. 

\subsection{Мультиплексор оповещений в разделяемой памяти}\label{chapter31:Mux}

С целью избежать излишних накладных расходов и использовать системные ресурсы наилучшим образом в настоящей работе предлагается метод мультиплексирования оповещений от множества соединений, использующих разделяемую память для передачи данных.

Мультиплексор оповещений в разделяемой памяти -- это структура данных, используемая множеством процессов-писателей для оповещения процесса-читателя о появлении данных в очереди в разделяемой памяти. Каждый процесс, участвующий в межпроцессном взаимодействии, должен иметь свой мультиплексор оповещений, через который ему будут поступать оповещения о появлении данных в разделяемой памяти, которые он может считать и обработать.

Время жизни мультиплексора оповещений определяется процессом-читателем. При необходимости процесс-читатель создает файл определенного размера в ФС и отображает его в свою память. Во время установления соединения процесс-читатель ассоциирует соединение с номером от 0 до 2047 и отправляет его вместе с путем до файла другой стороне. Эти данные используются противоположной стороной для отправки оповещений.

\textbf{TBD: нарисовать схему с двумя процессами и файлом, отображенным в их памяти}

\subsubsection{Структура и алгоритм работы мультиплексора оповещений в разделяемой памяти}\label{chapter31:MuxStructure}

Структура мультиплексора оповещений представлена на рисунке \ref{chapter31:MuxZeroState}. Он состоит из 4-байтного целого числа futex, используемого для синхронизации взаимодействующих процессов, и массив из 32 8-байтных сигнальных чисел, по одному на каждый бит futex. Эти 32 8-байтных числа содержат 2048 бит, что позволяет различать 2048 различных соединений. Описание на языке C++ представлено на листинге \ref{chapter31:MultiplexerStruct}.


\begin{figure}[!h]
\caption{Структура мультиплексора оповещений в разделяемой памяти}
\label{chapter31:MuxZeroState}
\includegraphics[width=\textwidth]{../../graphics/schemes/futex}
\end{figure}

\begin{lstlisting}[float=!h,caption={Структура мультиплексора в памяти},label={chapter31:MultiplexerStruct},frame=tlrb]
struct Multiplexer
{
    using Futex = std::atomic<int32_t>;
    using Signal = uint64_t;

    static constexpr std::size_t c_num_chunks = sizeof(Futex) * CHAR_BIT;
    static constexpr std::size_t c_signals_per_chunk = sizeof(std::atomic<Signal>) * CHAR_BIT;
	
	// Процедура ожидания на futex
    void wait() {
    	if (!m_futex) {
   			futex_wait(&m_futex, 0);
   		}
    }
	
	// Процедура оповещения процесса-читателя. Выставляет соответствующие биты мультиплексора для сигнала за номером signal и при необходимости пробуждает поток мультиплексора процесса-читателя.
    void notify(Multiplexer::Signal id);
    
    // Процедура для пробуждения потока, спящего на futex.
    void wakeup();

protected:
	// 4-байтное число futex, на котором происходит синхронизация сна/пробуждения потока мультиплексора.
    Futex m_futex;
	
	// Для избежания лишней состязательности между атомарными операциями над массивом сигнальных чисел и над futex, массив выровнен на размер кэш-линии процессора -- 64 байта.
    alignas(64) std::array<std::atomic<Signal>, c_num_chunks> m_signals;
};
\end{lstlisting}

Когда процесс-писатель хочет оповестить процесс-читатель о наличии данных в очереди в разделяемой памяти, ему необходимо:
\begin{enumerate}
\item атомарно выставить бит в сигнальном числе, соответствующий этому соединению;
\item атомарно выставить соответствующий бит futex и получить его предыдущее значение;
\item Если предыдущее значение равно нулю, значит, разбудить процесс-читатель, чтобы он мог обработать оповещение.
\item Если оно не равно нулю, это значит, что другой процесс-писатель уже либо разбудил целевой процесс-читатель, либо в скором времени сделает это (так как он был тем самым процессом, перед которым в futex был нуль).
\end{enumerate}

После завершения первых двух этапов мультиплексор для соединения под номером \#1987 будет находиться в состоянии, представленном на рисунке \ref{chapter31:Mux1987State}. Чтобы проставить нужные биты для данного, процесс-писатель делит номер его соединения на 64 и получает индекс бита в futex -- 31, и, соответственно, индекс сигнального числа. Далее он вычисляет остаток от деления номера сигнала на 64 и получает индекс бита в сигнальном числе. После чего атомарной операцией ''ИЛИ`` выставляется сначала нужный бит в сигнальном числе, а потом нужный бит в futex. Псевдокод алгоритмов процесса-писателя и читателя приведены на листингах \ref{appendix91:SignalCode} и \ref{appendix91:ReceiverCode}.

\begin{figure}[!h]
\caption{Состояние мультиплексора оповещений в разделяемой памяти с активным сигналом \#1987}
\label{chapter31:Mux1987State}
\includegraphics[width=\textwidth]{../../graphics/schemes/futexready}
\end{figure}

\subsubsection{Достоинства и недостатки}

Данный метод решает проблему, описанную в разделе \ref{chapter31:NaivePolling}, позволяя определять соединение-источник сигнала за время, не зависящее от количества соединений.

По сравнению с вышеописанным методом межпроцессного взаимодействия, использующим TCP для оповещения о появлении данных в разделяемой памяти, данное решение обладает следующими \textbf{достоинствами}:
\begin{enumerate}
\item Подавляющая часть работы происходит в пользовательском пространстве. В лучшем сценарии отправка и получение оповещения не задействуют системные вызовы. Оповещение происходит посредством двух атомарных операций.
\item Ядро ОС используется только пробуждения и засыпания процессов.
\end{enumerate}

Таким образом, в худшем случае совершается один системный вызов для засыпания процесса в ожидании оповещений и один для пробуждения процесса. Пробуждение потока влечет временные затраты как со стороны процесса-писателя -- на пробуждение потока, так и со стороны процесса-читателя -- на уход в состояние сна и временную задержку на постановку потока мультиплексора на выполнение. В то же время, ожидание оповещений в состоянии сна позволяет экономить ресурс процессора.

\textbf{Недостатком} данного метода является механизм управление файлом мультиплексора оповещений в файловой системе. Поскольку файлом управляет процесс, то в случае его некорректного завершения файл не будет удален и останется на всегда. Некорректное завершение может произойти по следующим причинам: крах процесса вследствие программного дефекта, при отправке сигнала \textit{SIGKILL} процессу, который в большинстве случаев приводит к немедленному завершению процесса.

Данный недостаток можно решить, используя более продвинутый механизм создания файлов в ФС. Например, делегировать эту работу отдельному процессу или группе процессов. В таком случае такой процесс может отслеживать состояние своих клиентских процессов и при прекращении их работы выполнять подчистку ресурсов.

\subsection{Методы обслуживания соединений}

Имея более совершенный механизм оповещения процессов о появлении данных в разделяемой памяти необходимо разработать также и метод обслуживания получаемых оповещений. В данном подразделе предложены четыре метода:
\begin{enumerate}
\item Синхронный -- существует единственный выделенный поток мультиплексора. Он занимается непосредственно обслуживанием соединений.
\item ''Полусинхронный/Полуреактивный`` -- существует единственный выделенный поток мультиплексора (полуреактивная часть). Он диспетчеризует обслуживание оповещений в пул потоков (полусинхронная часть) \cite{schmidt1995half}.
\item ''Лидер/Последователи`` -- в один момент времени только один поток-лидер отслеживает состояние мультиплексора в пассивном режиме. То есть, при отсутствии оповещений процесс-лидер находится в состоянии сна. При обнаружении оповещений он передает лидерство произвольному потоку из пула и переходит к обслуживанию оповещений \cite{schmidt1998leader}.
\item ''Лидер/Последователи`` -- в один момент времени только один поток-лидер активно отслеживает состояние мультиплексора. То есть, постоянно опрашивает мультиплексор. При обнаружении оповещений он передает лидерство произвольному потоку из пула и переходит к обслуживанию оповещений.
\end{enumerate}

\subsubsection{Синхронный метод обслуживания соединений}

В данном методе для обслуживания оповещений выделяется один поток, который обслуживает только соединения, использующие мультиплексор оповещений. Диграмма его работы представлена на рисунке \ref{chapter31:SyncMuxSchema}.
В отсутствие заявок поток мультиплексора находится в состоянии сна. Когда необходимо обработать соединение, ядро пробуждает этот поток и он, в свою очередь, выполняет алгоритм на листинге \ref{appendix91:ReceiverCode}, синхронно обрабатывая соединения, для которых были получены оповещения. Алгоритм выполняется до тех пор, пока есть оповещения для обслуживания.

Метод может быть полезен для систем с малым количеством соединений. Из-за своей простоты он может показать лучшую временную задержку на передачу данных, так как нет необходимости, как в других методах, диспетчеризовать обслуживание соединения в пуле потоков или выбирать новый поток-лидер перед обслуживанием данного соединения.

Однако, поскольку поток, обслуживающий мультиплексор, один, то когда он обслуживает текущее соединение, все остальные активные соединения простаивают. Данный метод далее в работе не рассматривается, так как не смотря на потенциально более низкую временную задержку на передачу данных при его использовании, он ограничен в количестве одновременно активных соединений.

\begin{figure}[!h]
\caption{Диаграмма синхронного обслуживания соединений выделенным потоком мультиплексора}
\label{chapter31:SyncMuxSchema}
\includegraphics[width=\textwidth]{../../graphics/schemes/SyncMuxSequence}
\end{figure}


\subsubsection{Метод обслуживания соединений ''Полусинхронный/Полуреактивный``}\label{chapter31:BlockingHSHA}
Описан в ранней работе автора \cite{GubarevFutex}.
В данном методе оповещения обслуживает пул потоков. Пул потоков получает задачи на обслуживание от выделенного потока мультиплексора оповещений. Применение шаблона проектирования сетевых приложений ''Полусинхронный/Полуреактивный`` \cite{schmidt1995half} к синхронному методу позволяет избежать простаивания активных соединений и обслуживать одновременно столько же соединений, сколько потоков выделено в пуле. Диаграмма работы метода представлена на рисунке \ref{chapter31:HSHAMuxSchema}.

\begin{figure}[!h]
\caption{Диаграмма диспетчеризации и обслуживания соединений по методу ''Полусинхронный/Полуреактивный``}
\label{chapter31:HSHAMuxSchema}
\includegraphics[width=\textwidth]{../../graphics/schemes/HSHAMuxSequence}
\end{figure}

Получение оповещений для соединения и обслуживание заявок для этого соединения происходит в разных потоках. Необходимо сохранить гарантию последовательного обслуживания заявок в соединении. То есть, в один момент времени заявки одного соединения обслуживает не более одного потока. Может произойти ситуация, при которой очередное оповещение для очередной заявки будет получено во время обслуживания текущей. В таком случае, поток из пула должен обслужить новую заявку после текущей, либо обслуживание должно быть делегировано пулу потоков после обслуживания текущей заявки. Первая стратегия предпочтительнее, так как имеет лучшую локальность по памяти и времени: один и тот же поток последовательно обслуживает заявки для одного и того же соединения при их наличии. В обоих случаях необходимо синхронизировать диспетчеризацию и непосредственно процедуру обслуживания соединения.

На листингах \ref{chapter31:DispatchHSHASession} и \ref{chapter31:ServeHSHASession} представлен исходный код процедур диспетчеризации и обслуживания соединений. Между собой процедуры синхронизированы механизмом взаимного исключения и набором состояний.

\begin{lstlisting}[float=!h,caption={Процедура диспетчеризации обслуживания соединения},label={chapter31:DispatchHSHASession},frame=tlrb]
void IMultiplexerListener::check_close_n_handle_send()
{
    std::unique_lock<std::mutex> lock(m_input_close_mutex);
    if (m_state == State::Closed) {
        return;
    }
    if (m_state == State::Handling) {
        m_state = State::KeepHandling;
        return;
    }
    m_state = State::Handling;
    lock.unlock();

    m_thread_pool->dispatch_async([this] { this->process_send() });
}
\end{lstlisting}

\begin{lstlisting}[float=!h,caption={Процедура обслуживания соединения в мультиплексоре оповещений},label={chapter31:ServeHSHASession},frame=tlrb]
void IMultiplexerListener::process_send()
{
    while (true) {
        this->handle_send();	

        std::unique_lock<std::mutex> lock(m_input_close_mutex);
      
        if (m_state == State::Closed) {
            lock.unlock();
            this->on_deferred_close();
            break;
        }
        else if (m_state == State::Handling) {
            m_state = State::Idle;
            break;
        }
        else { //m_state == State::KeepHandling
            m_state = State::Handling;
        }
    }
}
\end{lstlisting}

Процедура \textit{check\_close\_n\_handle\_send()} вызывается потоком мультиплексора для диспетчеризации обслуживания соединения. В зависимости от состояния соединения, выполняется один из трех сценариев.
\begin{enumerate}
\item Если соединение закрыто, то игнорировать оповещение из мультиплексора. Подчисткой ресурсов соединения занимается либо поток, закрывающий соединение, либо поток, обслуживающий соединение.
\item Если соединение уже сейчас обслуживается (или же обслуживание диспетчеризовано), то необходимо через состояние соединение заставить поток из пула произвести обслуживание еще раз.
\item Если же соединение в данный момент не обслуживается, то пометить его таковым и диспетчеризовать обслуживание в пуле потоков.
\end{enumerate}

Процедура обслуживания заявок в соединении \textit{process\_send()} выполняется потоком из пула. Перед завершением обслуживания соединения необходимо проверить, было ли получено новое оповещение (состояние соединения \textit{KeepHandling}). И если да, то повторить обслуживание. Состояние \textit{KeepHandling} необходимо, так как иначе может произойти утеря оповещения. А именно в ситуации, когда поток мультиплексора при получении оповещения видит, что соединение еще обслуживается (состояние \textit{Handling}), а поток из пула уже закончил фактическое обслуживание и вскоре поменяет состояние соединения на \textit{Idle}.

\begin{lstlisting}[float=!h,caption={Процедура закрытия соединения в мультиплексоре оповещений},label={chapter31:CloseHSHASession},frame=tlrb]
bool IMultiplexerListener::try_deactivate_listener()
{
    std::unique_lock<std::mutex> lock(m_input_close_mutex);
    if (m_state != State::Idle) {
        lock.unlock();
        m_mux->deregister_listener(this);
        return true;
    }
    m_state = State::Closed;
    return false;
}
\end{lstlisting}

Закрытие соединения также необходимо синхронизировать с его обслуживанием. Если соединение обслуживается прямо сейчас, нельзя освобождать занятые им ресурсы. Необходимо дождаться завершения обслуживания и только потом провести подчистку. В настоящей работе предлагается в такой ситуации выполнять закрытие потоком, который в данный момент обслуживание соединение. Процедура \textit{try\_deactivate\_listener()} проверяет состояние соединения: закрывать соединение и освобождать ресурсы можно только в состоянии \textit{Idle}, в противном случае соединение помечается как закрытое, чтобы поток из пула мог завершить процедуру закрытия после завершения обслуживания соединения. Исходный код приведен на листинге \ref{chapter31:CloseHSHASession}.

Данный метод позволяет улучшить качество обслуживания соединений, используя параллелизм обслуживания на уровне соединений. Но использование дополнительных потоков при обслуживании может негативно влиять на временную задержку на передачу данных. Диспетчеризация и постановка потока из пула на выполнение для обслуживания заявки может занимать существенное время.

\subsubsection{Метод обслуживания соединений ''Лидер/Последователи``}\label{chapter31:BlockingLF}

Как было сказано выше, в методе обслуживания соединений ''Полусинхронный/Полуреактивный`` заявки в соединении обслуживаются потоками из пула потоков. От приема оповещения о наличии данных в разделяемой памяти до обслуживания соединения происходит диспетчеризация и постановка на выполнение потока из пула для обслуживания заявки. Использование метода ''Лидер/Последователи`` \cite{schmidt1998leader} позволяет избежать второго слагаемого и, возможно, получить меньшую временную задержку на передачу данных.

В рассматриваемом методе, в отличие от предыдущего ''Полусинхронный/Полуреактивый``, отсутствует выделенный поток мультиплексора. Мультиплексированием и обслуживанием занимаются потоки из пула потоков. Это позволяет избежать накладных расходов на поддержание лишнего потока, синхронизацию между потоком мультиплексора и потоками пула, временной задержки на постановку на выполнение потока для обслуживания соединений \cite[41]{POSA2}. Диаграмма работы метода представлена на рисунке \ref{chapter31:LFMuxSchema}.

\begin{figure}[!h]
\caption{Диаграмма обслуживания соединений по методу ''Лидер/Последователи``}
\label{chapter31:LFMuxSchema}
\includegraphics[width=\textwidth]{../../graphics/schemes/LFMuxSequence}
\end{figure}

В каждый момент времени состояние мультиплексора опрашивает не более одного потока, потока-лидера. При наличии оповещений для обслуживания, поток-лидер диспетчеризует в пул потоков постановку на выполнение нового лидера. Сам же он перестает быть лидером и занимается обслуживанием соединений, для которых получены оповещения. Это требует внесения изменений в процедуру опроса мультиплексора и обслуживания соединений. Изменения представлены на листинге \ref{appendix91:LFReceiverCode}. Среди них:
\begin{enumerate}
\item Поток-лидер должен запомнить все соединения, которые ему необходимо обслужить.
\item Каждое из этих соединений необходимо пометить как \textit{Handling}. Метод \textit{should\_process()} приведен на листинге \ref{chapter31:LFShouldProcess}.
\item Поток-лидер диспетчеризует нового лидера.
\item После обслуживания соединений для полученных оповещений поток переходит в состояние ожидания лидерства.
\end{enumerate}

\begin{lstlisting}[float=!h,caption={Процедура подготовления соединения к обслуживанию в модели обслуживания соединений ''Лидер/Последователи``},label={chapter31:LFShouldProcess},frame=tlrb]
bool IMultiplexerListener::should_process()
{
    std::unique_lock<std::mutex> lock(m_input_close_mutex);
    if (m_state == State::Closed) {
        return false;
    }
    if (m_state == State::Handling) {
        m_state = State::KeepHandling;
        return false;
    }
    m_state = State::Handling;
    return true;
}
\end{lstlisting}

Второй шаг необходим, так как в противном случае новый лидер может получить новое оповещение для уже обслуживаемых предыдущим лидером соединений и тогда два потока будут параллельно обслуживать одно и то же соединение, что недопустимо.

В настоящей работе реализована вариация метода, в которой поток-лидер обслуживает соединения для всех обнаруженных оповещений. Такой подход может быть эффективен, когда число одновременно активных соединений мало. Недостаток такой реализации состоит в худшем использовании параллелизма на уровне соединений и накоплении временной задержки на передачу данных для соединений, оказавшихся в очереди на обслуживание.

\subsubsection{Метод обслуживания соединений ''Лидер/Последователи`` с активным ожиданием}\label{chapter31:NonBlockingLF}

\chapterconclusion

\textbf{TBD: Вывод по главе}


%% Макрос для заключения. Совместим со старым стилевиком.
\startconclusionpage

Целью настоящей работы являлось уменьшение временной задержки на передачу данных между процессами в пределах одного физического узла по сравнению с TCP путем разработки и применения более эффективных методов межпроцессного взаимодействия. Было выполнено сравнение методов межпроцессного взаимодействия и синхронизации. Исследованы подходы других авторов. Разработаны и реализованы новые методы межпроцессного взаимодействия. Проведено экспериментальное сравнение новых методов с разработанными автором ранее.

Результатом работы стало семейство новых методов межпроцессного взаимодействия. Передача данных в них осуществляется через очередь в разделяемой памяти. Оповещение о появлении данных в очереди осуществляется через разработанный и реализованный мультиплексор оповещений в разделяемой памяти. Он позволяет производить большую часть операций по оповещению в пользовательской памяти, а ядро использовать только при необходимости процессу-читателю ожидать оповещений в режиме сна и разбудить его процессу-писателю.

Были реализованы различные методы обслуживания соединений по полученным из мультиплексора оповещениям, исследовано влияние активного опроса мультиплексора на временную задержку на передачу данных. Исходя из полученных результатов, метод обслуживания соединений ''Лидер/Последователи`` при использовании с мультиплексором оповещений в разделяемой памяти показывает несущественно меньшую временную задержку на передачу данных, чем ''Полусинхронный/Полуреактивный``. Также, применение метода активного опроса мультиплексора позволило существенно уменьшить временную задержку на передачу данных по сравнению с аналогичным методом с пассивным ожиданием оповещений в мультиплексоре.

Основные результаты работы представлены на IX Конгрессе Молодых Ученых. Результаты работы использованы в программной платформе для высокочастотной алгоритмической торговли Tbricks компании Itiviti.

%В данном разделе размещается заключение.

\printmainbibliography

%% После этой команды chapter будет генерировать приложения, нумерованные русскими буквами.
%% \startappendices из старого стилевика будет делать то же самое
\appendix

\chapter{Псевдокод алгоритмов работы с мультиплексором событий в разделяемой памяти}\label{sec:app:1}

\begin{algorithm}[!h]
\caption{Псевдокод процедуры получения оповещений в мультиплексоре событий}
\label{appendix91:ReceiverCode}
\begin{lstlisting}[frame=tlrb]
void MutliplexerServer::handle_signals() {
	// Шаг 1. Если в futex записан 0, значит, нет оповещений для обработки. Тогда процесс переходит в состояние сна.
	m_mux->wait();
	// Шаг 2. Атомарно получить актуальное значение futex и установить вместо него 0.
	int32_t futex = atomic_exchange(&m_futex, 0);
	// Шаг 3. Подсчитать количество установленных битов в числе, чтобы не выполнять линейное сканирование всех 32 битов.
	uint8_t cnt = popcnt(futex);
	for (uint8_t i = 0; i < cnt; i++) {
		// Шаг 4. Для каждого бита futex проверить соответствующие ему сигнальные числа.
		uint8_t f = get_unset_lsb(&futex);
		// Шаг 5. Атомарно получить значение сигнального числа и записать в него 0.
		int64_t signal = atomic_exchange(&m_signal[f], 0);
		uint8_t nsignals = popcntl(signal);
		// Шаг 6. Для каждого найденного сигнала запустить его обработку.
		for (uint8_t j = 0; j < nsignals; j++) {
			uint8_t s = get_unset_lsb(&signal);
			this->handle_signal(i * 64 + s);
		}
	}
}

// Выполняет обработку соединения, которому ранее был выдан номер id
void MultiplexerServer::handle_signal(Signal id);

// Возвращает количество выставленных битов в числе
uint8_t popcnt(int32_t value);
uint8_t popcntl(int64_t value);

// Сбрасывает младший бит числа и возвращает его позицию.
uint8_t get_unset_lsb(uint32_t & value);
uint8_t get_unset_lsb(uint64_t & value);
\end{lstlisting}
\end{algorithm}

\begin{algorithm}[!h]
\caption{Псевдокод процедуры оповещения процесса}
\label{appendix91:SignalCode}
\begin{lstlisting}[frame=tlrb]
void Multiplexer::notify(Signal id) {
	// Шаг 1. Выставить нужный бит в одном из сигнальных чисел в массиве. Число находится как результат деления номера соединения на 64, т.е. число от 0 до 31, позиция бита как остаток от деления номера соединения на 64.
	m_signal[id / Multiplexer::c_signals_per_chunk].fetch_or(1 << id % Multiplexer::c_signals_per_chunk);
    // Шаг 2. Выставить бит futex, соответствующий сигнальному числу. Позиция нужного бита находится как результат деления нмоера соединения на 64, т.е. число от 0 до 31.
	uint32_t futex = m_futex.fetch_or(1 << id / Multiplexer::c_signals_per_chunk);
	if (!futex) {
		// Шаг 3. Если предыдущее значение futex было 0, то попытаться разбудить один процесс, спящий на futex, системным вызовом futex.
		this->wakeup();
	}
}
\end{lstlisting}
\end{algorithm}

%
%В приложениях рисунки, таблицы и другие подобные элементы нумеруются по приложениям с соответствующим префиксом. Проверим это.
%
%Листинг~\ref{lst4:apx} должен иметь номер А.1.
%
%\begin{algorithm}[!h]
%\caption{Исходный код и флоат \texttt{algorithm}}\label{lst4:apx}
%\begin{lstlisting}
%public class HelloWorld {
%    public static void main(String[] args) {
%        System.out.println("Hello, world!");
%    }
%}
%\end{lstlisting}
%\end{algorithm}
%
%Рисунок~\ref{fig2:apx} должен иметь номер A.1.
%
%\begin{figure}[!h]
%\caption{Пример рисунка}\label{fig2:apx}
%\centering
%\begin{tikzpicture}[scale=0.7]
%\draw[thick,->] (0,0)--(3.5,0);
%\draw[thick,->] (0,0)--(0,3.5);
%\draw[very thick, red] (0,0)--(3,3);
%\draw[dashed] (3,0)--(3,3);
%\draw[dashed] (1.5,0)--(1.5,1.5);
%\end{tikzpicture}
%\end{figure}
%
%Таблица~\ref{tab3:apx} должна иметь номер A.1.
%
%\begin{table}[!h]
%\caption{Таблица умножения с помощью \texttt{tabularx} (фрагмент)}\label{tab3:apx}
%\centering
%\begin{tabularx}{\textwidth}{|*{18}{>{\centering\arraybackslash}X|}}\hline
%-- & 1 & 2 & 3 & 4 & 5 & 6 & 7 & 8 & 9 & 10 & 11 & 12 & 13 & 14 & 15 & 16 & 17 \\\hline
%1  & 1 & 2 & 3 & 4 & 5 & 6 & 7 & 8 & 9 & 10 & 11 & 12 & 13 & 14 & 15 & 16 & 17 \\\hline
%2  & 2 & 4 & 6 & 8 & 10 & 12 & 14 & 16 & 18 & 20 & 22 & 24 & 26 & 28 & 30 & 32 & 34 \\\hline
%3  & 3 & 6 & 9 & 12 & 15 & 18 & 21 & 24 & 27 & 30 & 33 & 36 & 39 & 42 & 45 & 48 & 51 \\\hline
%4  & 4 & 8 & 12 & 16 & 20 & 24 & 28 & 32 & 36 & 40 & 44 & 48 & 52 & 56 & 60 & 64 & 68 \\\hline
%\end{tabularx}
%\end{table}
%
%Заодно проверим нумерованные и ненумерованные перечисления. Ненумерованные:
%\begin{itemize}
%    \item пункт А;
%    \item пункт Б;
%    \item пункт В.
%\end{itemize}
%
%Нумерованные списки нескольких уровней:
%\begin{enumerate}
%    \item первый элемент;
%    \item второй элемент с подэлементами:
%    \begin{enumerate}
%        \item первый подэлемент;
%        \item второй подэлемент;
%        \item третий подэлемент.
%    \end{enumerate}
%    \item третий элемент;
%    \item четвертый элемент;
%    \item пятый элемент;
%    \item шестой элемент;
%    \item седьмой элемент;
%    \item восьмой элемент;
%    \item девятый элемент;
%    \item десятый элемент.
%\end{enumerate}
%
%\chapter{Еще один пример приложения с неимоверно длиннющим названием для тестирования переносов}\label{sec:app:2}
%
%Проверим на примере таблиц, что нумерация в приложениях~--- по приложениям.
%Таблица~\ref{tab3:apx2} должна иметь номер Б.1.
%
%\begin{table}[!h]
%\caption{Таблица умножения с помощью \texttt{tabularx} (фрагмент)}\label{tab3:apx2}
%\centering
%\begin{tabularx}{\textwidth}{|*{18}{>{\centering\arraybackslash}X|}}\hline
%-- & 1 & 2 & 3 & 4 & 5 & 6 & 7 & 8 & 9 & 10 & 11 & 12 & 13 & 14 & 15 & 16 & 17 \\\hline
%1  & 1 & 2 & 3 & 4 & 5 & 6 & 7 & 8 & 9 & 10 & 11 & 12 & 13 & 14 & 15 & 16 & 17 \\\hline
%2  & 2 & 4 & 6 & 8 & 10 & 12 & 14 & 16 & 18 & 20 & 22 & 24 & 26 & 28 & 30 & 32 & 34 \\\hline
%3  & 3 & 6 & 9 & 12 & 15 & 18 & 21 & 24 & 27 & 30 & 33 & 36 & 39 & 42 & 45 & 48 & 51 \\\hline
%4  & 4 & 8 & 12 & 16 & 20 & 24 & 28 & 32 & 36 & 40 & 44 & 48 & 52 & 56 & 60 & 64 & 68 \\\hline
%\end{tabularx}
%\end{table}
%
%\chapter{Пример огромного листинга}
%
%\begin{lstlisting}[caption={Пример большого листинга},label={lstX}]
%import java.util.*;
%
%public class Example {
%    static int[] restoreOutgoing(int[] g, int[] outgoing,
%                                 int vertex, int mask) {
%        int[] rv = new int[1 + Integer.bitCount(mask)];
%        int n = g.length;
%        int current = rv.length - 1;
%        while (true) {
%            rv[current] = vertex;
%            if (current == 0) {
%                if (vertex != 0) {
%                    throw new AssertionError();
%                }
%                return rv;
%            }
%            mask ^= 1 << (vertex - 1);
%            int prevMask = outgoing[mask] & g[vertex];
%            if (prevMask == 0) {
%                throw new AssertionError();
%            }
%            vertex = Integer.numberOfTrailingZeros(prevMask);
%            --current;
%        }
%    }
%
%    static int[] restoreIncoming(int[] g, int[] incoming,
%                                 int vertex, int mask) {
%        int[] rv = new int[1 + Integer.bitCount(mask)];
%        int n = g.length;
%        int current = 0;
%        while (true) {
%            rv[current] = vertex;
%            if (current == rv.length - 1) {
%                if (vertex != 0) {
%                    throw new AssertionError();
%                }
%                return rv;
%            }
%            mask ^= 1 << (vertex - 1);
%            int nextMask = incoming[mask] & g[vertex];
%            if (nextMask == 0) {
%                throw new AssertionError();
%            }
%            vertex = Integer.numberOfTrailingZeros(nextMask);
%            ++current;
%        }
%    }
%}
%\end{lstlisting}
%                

\end{document}